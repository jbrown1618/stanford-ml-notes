\chapter{Week 2}

\section{Multivariate Linear Regression}

\subsection{Notation}

Notation revisited---we have to introduce a little bit of 
new notation to deal with multiple input features.

\begin{center}
  \begin{tabular}{l l}
    \hline
    Notation                    & Meaning \\
    \hline
    $m$                         & The number of training examples \\
    $n$                         & The number of variables \\
    $X$                         & The space of input variables \\
    $\xi$                       & The input features of the $i^{th}$ training example \\
    $\xij$                      & The value of the $j^{th}$ feature in the $i^{th}$ training example \\
    $Y$                         & The space of outputs \\
    $\yi$                       & The $i^{th}$ value of the output variable \\
    $\theta_j$                  & The $j^{th}$ model parameter \\
    $J(\theta_0, \theta_1)$     & The cost function \\
    $h_{\theta_0, \theta_1}(x)$ & The hypothesis function \\
    \hline
  \end{tabular}
\end{center}

\subsection{Multiple Features}

We also need a new form for the hypothesis function:

\[ h_{\theta}(x) = \theta_0 + \theta_1 x_1 + \theta_2 x_2 + ... + \theta_n x_n = \theta_0 + \sum_{j=1}^n \theta_j x_j \]

Alternately, define $x_0 = 1$.  So,

\[ h_{\theta}(x) = \sum_{j=0}^n \theta_j x_j \]

Or even more simply,

\[ h_{\theta}(x) = \vb{\theta} \cdot \vb{x} = \vb{\theta}^T \vb{x} \]

\subsection{Gradient Descent}

The only thing new is the partial derivative term:

\[ \pdv{\theta_j} J(\vb{\theta})  = \frac{1}{m} \sum_{i=0}^m \qty( h_{\theta}(x) - \yi ) \xij \]

\[ \pdv{\theta_j} J(\vb{\theta})  = \frac{1}{m} \sum_{i=0}^m \qty( \vb{\theta}^T \vb{x}^{(i)} - \yi ) \xij \]

So our gradient descent algorithm becomes

\[ \theta_j := \theta_j - \alpha \pdv{\theta_j} J(\vb{\theta})  \]

\[ \theta_j := \theta_j - \alpha \frac{1}{m} \sum_{i=0}^m \qty( \vb{\theta}^T \vb{x}^{(i)} - \yi ) \xij \]

\subsection{Feature Scaling}

Gradient descent converges more quickly when the features have a similar scale.
When the scales are very different, the contour ellipsoids are very narrow, so the gradient descent can oscillate.
Concretely, just get each variable to a $-1 \leq x \leq 1$ range.
Roughly.
You can also center the data.

\[ x_j := \frac{x_j - \mu_j}{\sigma_j} \]

\subsection{Learning Rate}

For a good value of $\alpha$, the cost $J$ should decrease with every iteration.
If it diverges or oscillates, then $\alpha$ is too large.
If it converges too slowly, then $\alpha$ is too small.
Ideally, find one value that is too large and one value that is too small, and choose something in between.

\subsection{Features and Polynomial Regression}

You can also feature engineer.

\begin{itemize}
  \item Combine features together to form new features
  \item Operate on a single feature for form a new feature (square it, etc.)
\end{itemize}

When you use polynomial features, feature scaling becomes especially important.

\subsection{Computing Parameters Analytically}

An alternative to gradient descent is using a closed-form solution which identifies the best fit.
When $X\theta = y$ is overdetermined, the best-fit solution is given by the Normal Equation:

\[ X^T X \theta = X^T y \]

\[ \theta = (X^T X)^{-1} X^T y \]

This is obtained by solving $\pdv{\theta_j}J_{\theta} = 0$ for all $j$.
In this context, $X$ is sometimes called the Design Matrix, and it is $m \times (n + 1)$.

Octave:

\begin{lstlisting}[style=Matlab-editor]
  theta = pinv(X' * X) * X' * y
  % pinv will give the best theta even if X' * X is not invertible!
\end{lstlisting}

Advantages of normal equation:

\begin{itemize}
  \item No need to choose $\alpha$
  \item No need to iterate
\end{itemize}

Disadvantages of normal equation:

\begin{itemize}
  \item Does not work as well for large $n$ (much bigger than 1000) 
    because $X^T X$ becomes an expensive calculation ($\mathcal{O}(n^3)$)
\end{itemize}

\section{Octave}

\begin{lstlisting}[style=Matlab-editor]
 % Range (inclusive)
 r = 1:6
 % [1 2 3 4 5 6]

 v = 1:0.1:2
 % [1.0 1.1 1.2 1.3 1.4 1.5 1.6 1.7 1.8 1.9 2.0]

 % Print a histogram
 normal = -6 + 10 * randn(1, 10000);
 bincount = 50;
 hist(normal, bincount)

 % Dimensions
 A = [1 2; 3 4; 5 6];
 length(A) % 3 - length gives the longest dimension
 size(A) % [3 2];

 % Load data
 load('datafile.dat')

 % Indexing
 y = randn(1, 100000)
 v = y(1:10) % v is the first ten elements of y

 A(3,2) % 6
 A(2,:) % [3 4]
 A(:,2) % [2;4;6]
 A([1 3], :) % [1 2; 5 6]
 A(:) % [1; 2; 3; 4; 5; 6]
 B = [A A] % [1 2 1 2; 3 4 3 4; 5 6 5 6]
 

 % Save variable to a file
 save filename.mat v
 clear
 % Restore v from the file
 load filename.mat

 % Use a cleartext format instead of binary
 save filename.txt v -ascii

 % Multiplication
 C = A * B

 % Elementwise Multiplication
 C = A .* B

 % Elementwise exponentiation
 C = A .^ 2

 % Elementwise division
 C = 1 ./ A

 % log, exp, abs, +, -, <, > are elementwise as well

 % Transpose
 AT = A'

 v = [1 3 2]
 A = [1 3 2; 4 2 5]

 [val, ind] = max(v) % [[3, 2]]
 ind = find(v < 3) % [1 3]
 [r, c] = find(A < 3) % [1 1 2], [1 3 2]

 % Heatmap of Matrix
 imagesc(A)

 %
 % Control flow
 %
 v = zeros(1, 10)
 for i = 1:10
     v(1,i) = i^2;
 end
   

indices = 1:10
for i = indices
    v(1,i) = i^3
end

j = 100
n = 0
while j > 1
    n = n + j
    j = j - 1
    if j == 53
        break
    elseif j == 63
        j = j - 4
    end
end
\end{lstlisting}

\subsection{Function files}

\begin{lstlisting}[style=Matlab-editor]
  % In squareAndCube.m
  function [y1, y2] = squareAndCube(x)
  y1 = x * x;
  y2 = y1 * x;
  end
\end{lstlisting}

To make it accessible, you can either change to its directory,
or use \texttt{addpath('/path/to/file/')}

\subsection{Vectorization}

\begin{lstlisting}[style=Matlab-editor]
  x = [1, 2, 3, 4, 5];
  theta = [2, 3, 4, 5, 6];

  % Slowly...
  val = 0;
  for i = 1:length(x)
      val = val + x(i) * theta(i)
  end

  % Alternately
  val = theta' * x
\end{lstlisting}

\[
  J = \frac{1}{2m} \sum \qty( X\theta - y)^T \qty( X\theta - y)
      + \frac{\lambda}{2m} \sum \theta \ewise \theta 
\]

\[
  \pdv{\theta} J = \frac{1}{m} X^T (X\theta - y)
                 + \frac{\lambda}{m} \theta
\]