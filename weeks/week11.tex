\chapter{Week 11}

\section{Case Study: Photo OCR}

\subsection{Problem Description and Pipeline}

OCR stands for Optical Character Recognition.
There are two steps: first, identifying where there is text in a picture,
then segmenting that text into characters,
and finally identifying the characters in that text.
This is not difficult for scanned documents, but is very difficult for photographs.

This sequence of steps, each of which uses ML, is an example of a pipeline.

\subsection{Sliding Windows}

Pedestrian detection is a simpler problem, because the aspect ratios of rectangles
around a person all have about the same size.
A neural network can be trained to take a window and identify whether or not it
includes a pedestrian.
Then the window is shifted, and the classification is repeated.
This entire process can be repeated with a variety of window sizes.

For an example like text detection, wherein the aspect ratios vary, we can still
use windows of a fixed size, but then apply an ``expansion operator'' to attempt
to combine adjacent regions of text.

Sliding windows can also be used for character segmentation; in this case, a neural
network can be trained to identify whether a window contains the midpoint between
two characters.

\subsection{Getting Lots of Data and Artificial Data}

For character recognition, you can use free font data to generate artificial data
to combine with real data.
You can also take real data and apply transformations (like distortion) to amplify the data set.
This technique works quite well for generating speech recognition sample data as well.
The distortions should be meaningful and representative of what might appear in the test set.

Before expanding the data set, make sure you have a low-bias classifier or it will not help!

``How much work would it be to get 10 times as much data as we currently have?''

\begin{itemize}
    \item Artificial data synthesis
    \item Collect and label it yourself
    \item ``Crowd source'' (Amazon Mechanical Turk)
\end{itemize}

\subsection{Ceiling Analysis: What Part of the Pipeline to Work on Next}

Having a single real-valued metric helps us to decide what to work on.

Ceiling analysis: given perfect inputs for each stage in the pipeline,
what is the impact on the overall system accuracy?
Whichever stage provides the largest jump is probably a good place to spend effort.
If having perfect text location data makes the overall system much better, spend time
on text identification!
